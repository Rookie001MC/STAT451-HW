\newpage
\section{Exercise 29:}
\begin{quote}
    As of April 2006, roughly 50 million .com web domain names were registered (e.g., yahoo.com).
\end{quote}

\subsection{ How many domain names consisting of just two letters in sequence can be formed? How many domain names of length two are there if digits as well as letters are permitted as characters? [Note: A character length of three or more is now mandated.]}


\begin{itemize}
    \item There are 26 letters in the English alphabet. Therefore, the number of domain names consisting of just two letters in sequence is \(26^2 = 676\).
    \item If digits are permitted as characters, then there are 26 letters and 10 digits. Therefore, the number of domain names of length two is \(36^2 = 1296\).
\end{itemize}

\subsection{How many domain names are there consisting of
three letters in sequence? How many of this length
are there if either letters or digits are permitted?
[Note: All are currently taken.]}

\begin{itemize}
    \item \(26^3 = 17576\).
    \item \(36^3 = 46656\).
\end{itemize}

\subsection{Answer the questions posed in 3.2 for four-character
    sequences.}

\begin{itemize}
    \item \(26^4 = 456976\).
    \item \(36^4 = 1679616\).
\end{itemize}

\subsection{As of April 2006, 97,786 of the four-character se    quences using either letters or digits had not yet been claimed. If a four-character name is randomly selected, what is the probability that it is already owned?}

Let P be the probability that a four-character name is not claimed.
\begin{equation}
    \begin{split}
        P & = \frac{97786}{1679616} \\
          & \approx 0.058
    \end{split}
\end{equation}
Therefore, the probability that a four-character name is already owned is \(1 - 0.058 = 0.942\).