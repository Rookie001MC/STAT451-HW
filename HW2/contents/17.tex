\newpage
\section{Exercise 17:}

\begin{displayquote}
    Let A denote the event that the next request for assistance from a statistical software consultant relates to the SPSS package, and let B be the event that the next request is for help with SAS\@. Suppose that \(P(A) = .30\)  and \(P(B) = .50\).
    \begin{itemize}
        \item [a.] Why is it not the case that \(P(A) + P(B) = 1\)?
        \item [b.] Calculate \(P(A')\).
        \item [c.] Calculate \(P(A \cup B)\).
        \item [d.] Calculate \(P(A' \cap B')\).
    \end{itemize}
\end{displayquote}

\subsection{a. Why is it not the case that \(P(A) + P(B) = 1\)?}
The events A and B are not mutually exclusive (disjointed), because, apart from the 2 package above, there are probably other packages that the consultant can use. Therefore, the sum of the probabilities of A and B is not equal to 1.

\subsection{b. Calculate \(P(A')\).}
\(P(A') = 1 - P(A) = 1 - 0.30 = 0.70\)

\subsection{c. Calculate \(P(A \cup B)\).}
Because A and B are disjointed, the intersection of A and B is 0:
\begin{equation}
    \begin{split}
        P(A \cup B) & = P(A) + P(B) - P(A \cap B) = 0.30 + 0.50 - P(A \cap B) \\
                    & = 0.80 - P(A \cap B)                                    \\
                    & = 0.80 - 0                                              \\
                    & = 0.80
    \end{split}
\end{equation}

\subsection{d. Calculate \(P(A' \cap B')\).}
Use De Morgan's Law:
\begin{equation}
    \begin{split}
        P(A' \cap B') & = P[(A \cup B)'] = 1 - P(A \cup B) = 1 - 0.80 = 0.20
    \end{split}
\end{equation}